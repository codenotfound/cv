%%%%%%%%%%%%%%%%%%%%%%%%%%%%%%%%%%%%%%%%%
% Plasmati Graduate CV
% LaTeX Template
% Version 1.0 (24/3/13)
%
% This template has been downloaded from:
% http://www.LaTeXTemplates.com
%
% Original author:
% Alessandro Plasmati (alessandro.plasmati@gmail.com)
%
% License:
% CC BY-NC-SA 3.0 (http://creativecommons.org/licenses/by-nc-sa/3.0/)
%
% Important note:
% This template needs to be compiled with XeLaTeX.
% The main document font is called Fontin and can be downloaded for free
% from here: http://www.exljbris.com/fontin.html
%
%%%%%%%%%%%%%%%%%%%%%%%%%%%%%%%%%%%%%%%%%

%----------------------------------------------------------------------------------------
%	PACKAGES AND OTHER DOCUMENT CONFIGURATIONS
%----------------------------------------------------------------------------------------

\documentclass[a4paper,10pt]{article} % Default font size and paper size
\usepackage{polyglossia}
\setmainlanguage{russian} 
\setotherlanguage{english}

% XeLaTeX can use any font installed in your system fonts folder
% Linux Libertine in the next line can be replaced with any 
% OpenType or TrueType font that supports the Cyrillic script.

\newfontfamily\russianfont[Script=Cyrillic]{Carlito}


\usepackage{fontspec} % For loading fonts
\defaultfontfeatures{Mapping=tex-text}
% \setmainfont[SmallCapsFont = Fontin SmallCaps]{Fontin} % Main document font

\usepackage{xunicode,xltxtra,url,parskip} % Formatting packages

\usepackage[usenames,dvipsnames]{xcolor} % Required for specifying custom colors

\usepackage[big]{layaureo} % Margin formatting of the A4 page, an alternative to layaureo can be \usepackage{fullpage}
% To reduce the height of the top margin uncomment: \addtolength{\voffset}{-1.3cm}

\usepackage{hyperref} % Required for adding links	and customizing them
\definecolor{linkcolour}{rgb}{0,0.2,0.6} % Link color
\hypersetup{colorlinks,breaklinks,urlcolor=linkcolour,linkcolor=linkcolour} % Set link colors throughout the document

\usepackage{titlesec} % Used to customize the \section command
\titleformat{\section}{\Large\scshape\raggedright}{}{0em}{}[\titlerule] % Text formatting of sections
\titlespacing{\section}{0pt}{3pt}{3pt} % Spacing around sections

\begin{document}

\pagestyle{empty} % Removes page numbering

\font\fb=''[cmr10]'' % Change the font of the \LaTeX command under the skills section

%----------------------------------------------------------------------------------------
%	NAME AND CONTACT INFORMATION
%----------------------------------------------------------------------------------------

\par{\centering{\Huge Дмитрий \textsc{Великий}}\bigskip\par} % Your name

\section{Персональные данные}

\begin{tabular}{rl}
    \textsc{Дата рождения:} & 28 Января 1992 \\
    \textsc{Адрес:} &  Санкт-Петербург, шоссе Революции, д. 37 корп. 2 кв. 77 \\
    \textsc{Телефон:} & +7 911 705-00-13\\
    \textsc{email:} &
    \href{mailto:dmitry.velikij@gmail.com}{dmitry.velikij@gmail.com}\\
    \textsc{Сайт-портфолио} &
    \href{http://velikiyweb.ru}{http://velikiyweb.ru}
\end{tabular}

%----------------------------------------------------------------------------------------
%	WORK EXPERIENCE 
%----------------------------------------------------------------------------------------

\section{Опыт работы}

%------------------------------------------------


\begin{tabular}{r|p{11cm}}
    \textsc{Сентябрь 2015-Январь 2016} & Веб-разработчик в составе команды
    ColdCrystal.ru\\
    & \footnotesize{Верстка (HTML, CSS, JS, Bootstrap), управление контентом
        (MODX), административные процессы (поиск клиентов, формирование
        коммерческих предложений)}\\
\multicolumn{2}{c}{} \\
    \textsc{Ноябрь 2014-Июнь 2015} & Инженер кафедры управления государственными
    информационными системами Университета ИТМО\\
    & \footnotesize{Администрирование информационных систем на базе Linux,
        работа с открытыми данными (CKAN), реструктуризация архива дел кафедры,
        редактирование сборников статей конференций}\\
\multicolumn{2}{c}{} \\
\textsc{Март 2015} & Научно-исследовательская практика в Санкт-Петербургском
информационно-аналитическом центре\\
& \footnotesize{Разработка и программная реализация метода прогнозирования временных рядов на основе нечеткой
    логики для решения задач аналитики социальных процессов}\\
\multicolumn{2}{c}{} \\

%------------------------------------------------

\textsc{Август 2013} & Производственная практика в Комитете по экономической политике и
стратегическому планированию Санкт-Петербурга \\
& \footnotesize{Помощь в работе над составлением документов, используемых при
    создании
    планов и прогнозов социально-экономического развития города, в том числе выполнения Указов Президента РФ.}
\end{tabular}

%----------------------------------------------------------------------------------------
%	EDUCATION
%----------------------------------------------------------------------------------------

\section{Образование}

\begin{tabular}{rl}	
    \textsc{Июнь} 2015 & Магистр по направлению "Прикладная информатика",
        \textbf{Университет ИТМО}, Санкт-Петербург\\
        & специальность: \small\emph{"Управление государственными информационными
            системами"}
\medskip
\\
& Тема диссертации: "Поддержка принятия управленческих решений для развития
\\&
урбанизированных территорий на примере противодействия распространению
\\&
наркомании в Санкт-Петербурге" | \small Научный руководитель:  к.т.н.
\textsc{Захаров} Юрий
Никитич \\

\multicolumn{2}{c}{} \\
%------------------------------------------------

\textsc{Июнь} 2013 & Бакалавр по направлению "Прикладная информатика",
\textbf{СПБГУ}, Санкт-Петербург\\
& специальность: \small\emph{"Прикладная информатика в области искусств и
    гуманитарных наук"}
\end{tabular}

%----------------------------------------------------------------------------------------
%	SCHOLARSHIPS AND ADDITIONAL INFO
%----------------------------------------------------------------------------------------

\section{Стипендии и сертификаты}

\begin{tabular}{rl}
    \textsc{Сентябрь} 2008-Январь 2010& Стипендия программы "Талантливая
    молодёжь России" \\&Благотворительного Фонда "Русский Стандарт"\\
\multicolumn{2}{c}{} \\
    \textsc{Март} 2009& Первый Кембриджский сертификат, оценка "B"
\\
\end{tabular}

%----------------------------------------------------------------------------------------
%	LANGUAGES
%----------------------------------------------------------------------------------------

\section{Языки}

\begin{tabular}{rl}
    \textsc{Английский:} & Свободное владение\\
    \multicolumn{2}{c}{} \\
    \textsc{Немецкий:} & Базовый уровень\\
\end{tabular}

%----------------------------------------------------------------------------------------
%	COMPUTER SKILLS 
%----------------------------------------------------------------------------------------

\section{Профессиональные навыки}

\begin{tabular}{rl}
    Продвинутые знания:& веб-разработка (HTML5, CSS3, Javascript, SASS, jQuery,
    Bootstrap, \\&Handlebars, MODX, Drupal, Plone), \\&анализ данных (R, Python), 
    \\&работа с открытыми данными (web-API, scraping, CKAN), \\&системный анализ (UML), \\&прикладное программирование, работа в IDE
    (vim, tmux, gdb).\\
\multicolumn{2}{c}{} \\
Базовые знания: & основы управления IT-проектами, Linux, MySQL, PHP, Java, C++, {\fb \LaTeX}\\
\end{tabular}

\end{document}
